
%%%%%%%%%%%%%%%%%%%%%%% file typeinst.tex %%%%%%%%%%%%%%%%%%%%%%%%%
%
% This is the LaTeX source for the instructions to authors using
% the LaTeX document class 'llncs.cls' for contributions to
% the Lecture Notes in Computer Sciences series.
% http://www.springer.com/lncs       Springer Heidelberg 2006/05/04
%
% It may be used as a template for your own input - copy it
% to a new file with a new name and use it as the basis
% for your article.
%
% NB: the document class 'llncs' has its own and detailed documentation, see
% ftp://ftp.springer.de/data/pubftp/pub/tex/latex/llncs/latex2e/llncsdoc.pdf
%
%%%%%%%%%%%%%%%%%%%%%%%%%%%%%%%%%%%%%%%%%%%%%%%%%%%%%%%%%%%%%%%%%%%


\documentclass[runningheads]{llncs}
\usepackage{amsmath}
\usepackage{mathtools}
\usepackage{amssymb}
\usepackage{float}
\usepackage{multicol}

\setcounter{tocdepth}{3}
\usepackage{graphicx}
\usepackage[sorting=none]{biblatex}
\addbibresource{references.bib}



\usepackage{url}
\urldef{\mailsc}\path||
\newcommand{\keywords}[1]{\par\addvspace\baselineskip
\noindent\keywordname\enspace\ignorespaces#1}

\begin{document}

\mainmatter  % start of an individual contribution

% first the title is needed
\title{Can communication topology improve a multi-swarm PSO algorithms?}
% Los titulos no llevan punto al final - M
% Maybe this is a bit too long? Also, it would be best to say *why*
% you did this, instead of what you did - JJ

% a short form should be given in case it is too long for the running head
\titlerunning{Can comm. topo. improve multi-swarm PSO algs.?}

% the name(s) of the author(s) follow(s) next
%
% NB: Chinese authors should write their first names(s) in front of
% their surnames. This ensures that the names appear correctly in
% the running heads and the author index.
%
\author{José Guzmán \and Mario García-Valdez \and Juan J. Merelo-Guervós}

\authorrunning{Guzmán et al.}
% (feature abused for this document to repeat the title also on left hand pages)

% the affiliations are given next; don't give your e-mail address
% unless you accept that it will be published
\institute{Tijuana Institute of Technology, Tijuana, Mexico
  \email{jose.guzmanc19@tectijuana.edu.mx, mario@tectijuana.edu.mx} \and
University of Granada, Spain, \email{jmerelo@ugr.es}}



%
% NB: a more complex sample for affiliations and the mapping to the
% corresponding authors can be found in the file "llncs.dem"
% (search for the string "\mainmatter" where a contribution starts).
% "llncs.dem" accompanies the document class "llncs.cls".
%

\toctitle{}
\tocauthor{José Guzmán, Mario García-Valdez}

\maketitle


\begin{abstract}
  
  Using multiple-swarm PSO is a technique used in recent years to help improve
  the performance of nature-inspired optimization algorithms. A distributed PSO
  algorithm can work in every swarm in parallel and asynchronously communicate
  particles between them. However, the communication design is not a trivial
  task because any architectural change will affect how it explores the search
  space and how it exploits the best particles. Nevertheless, it has been
  reported that the exchange of possible solutions helps optimizations
  algorithms to improve their performance. This paper focuses on proposing and
  comparing two communication policies regarding how the communication graph
  between swarms is organized. These policies intend to limit the communication
  between populations to increase per-swarm exploration and avoid premature
  convergence through exploring different parts of the space in different
  swarms. The proposed policies are chain and hypercube topologies and compare
  them against a simple cross-over strategy using several continuous
  optimization benchmark functions to assess the benefits of choosing one
  communication topology over another. After the experiments, the chain-based
  topology had a better error performance.


  %Using multiple-swarm PSO is a technique used in recent years to help improve
  %the performance of nature-inspired optimization algorithms. A distributed PSO
  %algorithm can work in every swarm in parallel and also communicate particles
  %between them asynchronously. However, the design of the communication between them    % Which aspect of
                                % the design? - JJ
  %is not a
  %trivial task because any architectural change will affect how the
  %search space is explored, and how better solutions found are
  %exploited. % Maybe add a sentence talking about how communication
             % topology has a big influence - JJ
  %In this paper, we focus on proposing and 
  %comparing two communication policies regarding how the communication
  %graph between swarms is organized. These policies intend to limit the communication between
  %populations to increase per-swarm exploration
  %and avoid premature convergence % via the exploration of different
                                % parts of the space in different swarms.
  %. The
  %proposed policies are chain and hypercubic topologies. % Why only
                                % these two? - JJ
  %We implemented them in an event-based cloud-native design. We
  %compared the three options % two options above and which other one?
                             % - JJ
  %using several continuous optimization benchmark functions to assess the
  %benefits of choosing one communication topology over another. After the experiments, the
  %chain topology had a better performance using MSE as a metric.
  % Why do you mention only MSE metric at the end? Are there other
  % metrics? - JJ
  
  % rewritten above - M  


\keywords{multi-swarm intelligence, communication topologies, multi-swarm PSO.}
\end{abstract}


\section{Introduction}

% 1. State the context
% 2. State the problem
% 3. Say the hypothesis you want to prove in the paper, its objective.
% 4. Say how you want to prove that hypothesis.

Bio-inspired optimization algorithms are intrinsically parallel, and researchers
have exploited this fact since their conception. A typical implementation is to
divide a global population into many smaller populations working in parallel and
(possibly) asynchronously. Multipopulation designs have been used to solve
large-scale complex optimization problems, where the search space has many
locally optimal solutions \cite{a1}. Several studies are showing that
multi-population optimization often outperforms single-population approaches
\cite{b11} \cite{b12}. Another advantage of multi-population optimization is the
ability to have a parallel execution of each population, significantly improving
computation time. Many proposals are using multi-threaded, parallel, and
distributed processing to increase the speedup of the execution time \cite{b13}
\cite{b14}.

% Among the various strategies that have emerged in recent times for bio-inspired
% optimization algorithms, the creation of multiple populations working
% in parallel and (possibly) asynchronously has proven to be valuable in 
% solving complex optimization problems \cite{a1}.
% Researchers often divide the original population into small sub-populations in order to 
% achieve a certain outcome, for instance, solving large-scale optimization problems and dynamic
% optimization problems. There are some studies on multi-population optimization that show
% that it is possible integrate this aproach easily into various nature-inspired optimization algorithms,
% and often outperforms the ones using single-population optimization algorithms \cite{b11} \cite{b12}.
% Other investigations, reviews, and surveys on multi-population approaches have also
% been published in recent years, in which multi-population concepts are described
% using other terms like 'parallel,' 'Cooperative,' 'co-evolution,' and 'islands.'
% Its multi-threaded and parallel processing features that significantly speed up
% deployment time \cite{b13} \cite{b14}. % Please rewrite - JJ
% rewritten above - M 

One of the critical points in this technique is communication between
populations, since it has been observed in various studies that the
exchange of possible solutions helps optimizations algorithms to improve their 
performance \cite{a2}. Communication has different aspects:

\begin{itemize}
    \item A {\em speed} that defines how many solutions are able to be exchanged between sub-populations.
    \item A {\em policy} in charge of selecting which solutions should
      be replaced by those of coming from another sub-populations.
    \item An {\em interval} or gap that establishes the frequency with
      which interchanges between populations take place.
    \item A {\em connection topology} that specifies the exact way
      subpopulations communicate.
\end{itemize}


% Several distributed architectures have been proposed to manage the communication
% and parallel execution of these kind of algorithms, for example there is a
% similar approach by Scott Haberson et al. using a multi-population parallel
% genetic algorithm for independent evolution \cite{da1}. % This should
% % probably go to the state of the art.
% Moving it - M

% You need to say here why speed, policy and interval are not used in
% this paper - JJ
In this work, we focus only on the connection topology aspect of the 
communication, because    

In this paper, we propose the two communication topologies to be
implemented in a even-based distributed multi-swarm Particle Swarm
Optimization (PSO) algorithm.  These communication topologies follow a
chain and a hypercube structure on a multi-swarm. We chose these
topologies for their ability to be adapted to the architecture
mentioned above. They can work without altering the optimization
algorithm allowing a fair comparison with other communication methods.

This paper contains the following sections: We present
state-of-the-art on multi-swarm optimization in Section
\ref{sec:soa}. In Section 3, we present two variants of communication
topologies, to be used in an event-based architecture we also present
in the section. In Section 4, we describe the experimental setup and
results; additionally, Section 5 presents the conclusions for this
work.

\section{State of the Art}
\label{sec:soa}

Particle Swarm Optimization (PSO) is an optimization algorithm
inspired by the collective behavior of some flocks of birds and schools of fish. It was introduced in 1995 by  J. Kennedy and
R. Eberhart, and since then it has undergone several improvements \cite{b1}. % pon el nombre del autor - M -j
Since then, researchers have created different versions, aimed at different
purposes, developed new applications in various areas, published many
studies on the effects of various parameters, and proposed great
variety algorithm variants \cite{b2}. % This sentence does not really
                                % contribute to the paper. It's too
                                % generic, and does not have a
                                % point. - JJ
A basic version of the PSO algorithm
works with a population, called a swarm, of possible solutions, which
are denominated particles. They ``move'' in the search space according to
some simple rules derived from formulas. % This is not the state of
                                % the art, but a general introduction
                                % to the technique. Should go to the
                                % introduction, or better yet, just
                                % eliminate it completely. - JJ

The movements in the particles' search space are
determined by their best known locations and by the best known location
of the entire swarm. % Using "known" here is confusing. - JJ
These locations are supposed to get better and they will
guide the swarm's movements. The process is repeated many times, so it
is expected that solution that meets the requirements will eventually be
discovered, although this may not happen\cite{b3}. % All this should
                                % probably go to the intro.


% There should be a transitional sentence that goes from PSO to
% multi-swarm PSO. - JJ
Multi-swarm optimization is one of the most widely known variants of Particle Swarm Optimization
(PSO), and it is based the creation of multiple swarms (or sub-swarm)
instead of just one. % What is the motivation for MSPSO? Why do you
                     % focus on it on this paper? - JJ
The basic flow in
a multi-swarm optimization is that each sub-swarm has its own specific region to concentrate. A specific diversification method decides where and
when to locate and execute this sub-swarms. % First mention of
                                % diversification method. What is it?
                                % - JJ

% You should relate these to the challenges in single-swarm PSO and
% how they're approached in these different techniques. A SoA is not
% an enumeration of papers - JJ
For example, Wave of Swarm of Particles \cite{b6} uses a technique on
the "collision" of particles. % Technique on? Please rephrase - JJ
When the
particles are close to each other, they are sent into new sub-swarms,
preventing full convergence between the sub-swarms. Another examples is the Dynamic Multi-Swarm-Particle
Swarm Optimizer\cite{b7} that rearranges the particles from the sub-swarms
(when converged) into new ones periodically, so the new
iteration of sub-swarms have the advantage of starting with particles from the previous
one. Locust swarms \cite{b8} are founded on a "devour and go" strategy
after a sub-swarm consumes or "devours" a small fraction of the search space an explorer is deployed to search for
new regions making the sub-swarm move to the new promised location ("keep going").

In contrast to typical PSO swarms, sub-swarms are fed with information about
previous swarms. % previous in what sense? - JJ
These could be the positions and velocities instead of having
their initial parameters to be randomly selected. Generally speaking, the development of
multi-swarm systems creates a new path of design options that who were not present at the time the original PSO emerged. 
These design decisions now have guidelines thanks to the numerous studies on the topic, for
example, tow common options are the use of non-random initial positions and initial velocities for the particles in the subs-warms helping to
better results, which is not the case for individual swarms \cite{b9}.

A few number of this options can be reached by relatively independent
subcomponents that allow different approaches to this matter to be
tested. For example, the multi-swarm system UMDA-PSO \cite{b10} 
applies a multi-swarm hybrid approach using a diverse combination of 
elements from particle swarm optimization, distribution estimation 
algorithm, and differential evolution.

% Rewrite this too -- JJ
There have been some papers focused in optimizing the communication
between swarms, for example the survey in reference \cite{b15} has a
collection of different investigations around this subject. For
example El-Abd and Kamel talked over the multiple factors that can
change the overall behavior of multiple joined swarms, some of this factors
were the strategy and path of communications between swarms and the number sub-swarms. 
Then an experiment was made applying a circular topology of communications, 
the results demonstrated that this approach has an overall superior performance 
than using a simpler technique of sharing the global best of all the swarms\cite{b16}. 

Another example and the source of inspiration for this paper, is one
of Li and Zeng, % please check name spelling - JJ
who presented a multi-population algorithm with a chain-like structure for parallel global
numerical optimization. In this approximation a few changes where applied, like a dynamic neighborhood,
in order to improve the parallel optimization\cite{b17}.

% A state of the art section must have a conclusion, saying why your
% methods imply an advance over the state of the art - JJ

In the next section, we present two variants of communication topologies,
to be used in an event-based architecture for the implementation of 
distributed population-based optimization algorithms.


\section{Proposed method}

Recently, there is an interest in cloud-native, event-based architectures
suitable to run locally on a personal computer or in a cloud platform service.
An event-based architecture uses events to trigger and communicate between
services and processes. Using an event-based architecture, we intend to carry
out a series of experiments to analyze the effect of changing the communication
topology in a multi-population optimization algorithm's performance. Using this
type of architecture with a multi-swarm optimization algorithm is a new
approach. Therefore, there is an opportunity to find a substantial variation in
the performance of such an optimization algorithm by changing the way the
various swarms communicate to exchange information.

\subsection{EvoSwarm}

First, we briefly present the overall architecture of EvoSwarm, the
platform in which we implemented the distributed algorithm.

EvoSwarm follows an event-based architecture, using message queues for
interprocess communication. Messages consist of small populations or swarms that
are consumed by worker processes that run a local PSO for a small number of
iterations. After this, the resulting swarm is pushed to another message queue
to be consumed by a migration process responsible for the communication
(migration) between swarm messages kept in a buffer \cite{b18}.
% Better a higher-level description, describing what's in
            % the queues. Also, message queues. - JJ

            % (See Figure \ref{fig:example}).

% \begin{figure}[h]
% \centering
% \DeclareGraphicsExtensions{.jpg}
% \includegraphics[scale=.40]{Resources/F1-1.jpg}
% \caption{The EvoSwarm flow.}
% \label{fig:example}
% \end{figure}
% Eliminated. It's in Spanish...
% Better translate it ... it needs a figure - JJ

It is worth mentioning that this architecture can be implemented using various
computers in a network or using virtualization, in work we used Docker containers. 
Here are some of the most important features that
EvoSwarm brings to the user:

\begin{itemize} 
  
  \item Fully capable of scalability, since more computers can be added to the
  network in order to manage a more significant workload.
    
    \item Independence between processes, which means that any process in the
    structure is completely separated from the other ones allowing more work to
    be done at the same time.
    
    \item It is adaptable to any population-based optimizing algorithm.
\end{itemize}

The only change made to the original version of EvoSwarm \cite{b18} was the
communication policy between the populations of the multi-swarm PSO. % You should say why this change was made; also if it's been released as open source - JJ

\subsection{Proposed communication topologies for EvoSwarm}

In the original communication policy, the architecture % An
                                % architecture can't wait for
                                % anything. Also add a reference - JJ

waited until a minimum of
three populations reached the migration component. % Component? it
                                % would help if there was a figure - JJ
These populations are sent to
a process that combines the best half of each one with the others. For example,
let us name the three populations A, B, and C, then a sort method is applied to
each one (based on the quality of the solution). The sorted populations are now
divided into halves in preparation for the merging process, now, with three new
populations, the first one has the best half of A and B, the second has the best
of B and C, and the third has the best of A and C. The last step is reinserting
these new populations into a queue from which the PSO processes will retake
them.

The first modification, in this case, is only on the merging process. To create
a chained algorithm that affects every three arriving populations, we sort the
three populations. The modification consists in that the populations are only
allowed to share a few members of the elite. In this case, 10. Individuals are
carried into a chain structure in which a population can receive only one way of
the structure and shares in another.

The 2nd variation for the EvoSwarm structure is a hypercube; we based this one
on a paper in which this topology was applied to solve a multi-reservoir of
water using a multi-population algorithm \cite{b20}. The name indicates this
method has a cubic structure and can only function with eight populations or
more. For this paper, we only need eight populations. Once the algorithm gets
filled with eight populations, the migrations algorithm needs to assign each one
a place in the hypercube structure. Figure 3 it has shown the layout of the
structure.

The structure resembles a cube, and in every one of its vertices is one
population. One thing that stands immediately is the capital P because only four
populations have it. In this algorithm, we divided the hypercube into two
dimensions. Moreover, depending on the iteration of the experiment, the exchange
of solutions is restricted.

Every iteration of an experiment changes the mode in which the populations
communicate with each other. For example, in the first iteration, the eight
populations can only communicate with the others in the same dimension. In the
second iteration, the opposite happens, allowing them to exchange information
with a population outside that dimension. With eight vertices, only two
exchanges are allowed, and for the experiments on this paper, only the best 10\%
of solutions migrate to another swarm.


\section{Experimental Setup and Results}

We implemented the migration policies in Python; for the PSO algorithm, we use
the Evolopy library \cite{b19}, as mentioned earlier in this paper, we use the
same parameters for the PSO algorithm as the EvoSwarm paper (see Table 1).



\begin{table}[h!]
\centering
\caption{Parameters PSO of Evolopy library}
\begin{tabular}{|c c|} 
 \hline
 Parameters & Values  \\ [0.5ex] 
 \hline\hline
 Vmax & 6 \\ 
 Wmax & 0.9 \\
 Wmin & 0.2 \\
C1 & 2 \\
C2 & 2 \\[0.5ex]
 \hline
\end{tabular}
\label{table:1}
\end{table}

Continuing with the configuration of the experiments, Table 2 presents the
parameters used in EvoSwarm; these are for the 3 cases presented in this paper.

\begin{table}[h]
\centering
\caption{Parameters for EvoSwarm}
\scalebox{0.90}{
\begin{tabular}{|c c c c c|} 
 \hline
 Dimensions & Generations  & Population size & Num. Experiments & Num. Population created\\ [0.2ex] 
 \hline\hline
 10 & 50 & 70 & 30 & 10 \\ 
 20 & 66 & 100 & 30 & 10 \\[0.2ex]
 \hline
\end{tabular}}
\label{table:1}
\end{table}

\subsection{Benchmark Functions}
%Aqui poner ref del pdf de coco

In order to run the optimization experiments, we selected ten functions from the
COCO benchmark. We selected these functions because in our preliminary
experiments, they showed performance variations as we changed the communication
topology. Here are the ten functions:

\begin{multicols}{2}
\begin{itemize}

    \item Function 1: Sphere.
    \item Function 2: Ellipsoidal separable.
    \item Function 3: Rastrigin separable.
    \item Function 9: Rosenbrock rotated.
    \item Function 10: Ellipsoidal.
    \item Function 15: Rastrigin.
    \item Function 17: Schaffer F7, condition 10.
    \item Function 18: Schaffer F7, condition 1000.
    \item Function 21: Gallagher 101 peaks.
    \item Function 22: Gallagher 21 peaks

\end{itemize}
\end{multicols}

The number given to each function comes from COCO. This means that any method
using the benchmark can be compared against a great variety of optimization
methods. The complete collection of functions, graphics, and equations can be
reviewed in reference \cite{bbob}.

% Add some information about how the experiments were performed, type
% of machine, how much time they took ... -JJ

\subsection{Experimental Results}

We run the experiments in a single computer using a Ryzen 5 2600x CPU and 24 GB
of ram in Windows 10 Professional 64 bits, using Docker Desktop version 2.5,
running ten containers. We used Python version 3.5.7. We conducted each
experiment separately 30 times. % Are source and results available? - JJ
The time required for each run took about 7
hours. In these experiments, we are not interested in the total time required to
find a solution as the changes in the algorithm did not affect the running time.
In this case, we are focused on the MSE of the best solution found with the same
number of function evaluations.

In ten dimensions, there are three functions in which the original topology
gives better results than the topologies we propose (shown in boldface).
However, once we reach twenty dimensions, the original topology obtains the
worst results. Overall the chain topology gives a better MSE in 20 dimensions,
only having two cases with a lower MSE than the hypercube configuration. We can
see that the standard deviation of the MSE obtained by the three options only
varies in few cases. In particular, the function $f_{10}$ was difficult for this
algorithm and could not find a target in any run.

In the same table, we present the results of the statistical tests of comparing
both topologies against the original migration. In this case, we use a \(Z\)
statistical test with an $\alpha$ of 0.5, giving a critical value of 1.64.
Underlined results have sufficient statistical evidence to have less MSE; this
means that changing the communication topology can improve, in some cases, the
performance of a multi-population optimization algorithm.


\begin{table}[h!tb]
    \centering
    \caption{Comparison between the MSE obtained from 30 runs. We show the best results in boldface. Also, we
     present the statistical comparison against the EvoSwarm method,
     with results underlined if they are significantly better than the
     rest. }
    \scalebox{0.72}{
    \begin{tabular}{|c|c||c|c|c||c|c|c|}
    \hline
     Dimensions & Function & EvoSwarm MSE & Chain MSE & Hypercube MSE & EvoSwarm Stdev & Chain Stdev & Hypercube Stdev \\ [0.5ex] 
    \hline\hline
10	&	1	  &	4.78611E-09	          &	\textbf{4.45791E-09}	&	4.52861E-09	&	3.3503E-09	&	3.27217E-09	&	3.25265E-09	\\
10	&	2	  &	\textbf{3.83054E-09}	&	4.1792E-09	&	4.65155E-09	&	2.35352E-09	&	2.86765E-09	&	2.9481E-09	\\
10	&	3	  &	2.352927573	          &	 \underline{\textbf{0.537649443}}	&	\underline{0.805637277}	&	2.881847604	&	1.543574882	&	1.663900202	\\
10	&	9	  &	0.49249742	          &	0.2527938	&	\textbf{0.229809004}	&	0.983944824	&	0.152990827	&	0.109927138	\\
10	&	10	&	\underline{\textbf{250.1738327}}	&	430.5420728	&	374.4359744	&	228.3598149	&	527.2651437	&	323.3050765	\\
10	&	15	&	14.66500256	          &	\underline{\textbf{12.05469416}}	&	13.33777036	&	6.840266769	&	5.391630829	&	5.860112176	\\
10	&	17	&	0.0923583	&	\textbf{0.075762972}	&	0.087059177	&	0.16646265	&	0.155990531	&	0.17433065	\\
10	&	18	&	0.524852914	&	0.506412719	&	\textbf{0.413659866}	&	0.633592264	&	0.467805944	&	0.382221237	\\
10	&	21	&	0.329852279	&	0.237422927	&	\textbf{0.20667313}	&	0.621910649	&	0.481633605	&	0.390914491	\\
10	&	22	&	0.692064149	&	\textbf{0.514782111}	&	0.57179096	&	0.577929599	&	0.719118705	&	0.650216016	\\
20	&	1	&	0.03594249	&	5.55231E-09	&	\textbf{5.54663E-09}	&	0.196865099	&	2.9048E-09	&	2.52356E-09	\\
20	&	2	&	5.64673E-09	&	5.49755E-09	&	\textbf{5.36566E-09}	&	2.31813E-09	&	2.54807E-09	&	2.98753E-09	\\
20	&	3	&	12.082746	&	8.746638682	&	\underline{\textbf{7.874205319}}	&	8.54691241	&	7.803681637	&	7.808562704	\\
20	&	9	&	\textbf{11.10349719}	&	11.37025195	&	11.49485451	&	1.483577297	&	2.321860001	&	1.0945346	\\
20	&	10	&	4994.888993	&	\textbf{4472.971044}	&	5805.968994	&	4632.673198	&	2316.255771	&	5624.624998	\\
20	&	15	&	58.88102471	&	\textbf{53.54574023}	&	56.38870042	&	20.44460373	&	16.99626952	&	20.85090795	\\
20	&	17	&	0.788958338	&	0.766878619	&	\textbf{0.764906251}	&	0.43605249	&	0.434587453	&	0.487783929	\\
20	&	18	&	2.533386188	&	\textbf{2.255266883}	&	2.559019459	&	1.013613907	&	0.930554736	&	1.496684622	\\
20	&	21	&	0.901706552	&	\underline{\textbf{0.39256696}}	&	\underline{0.395204608}	&	1.040801093	&	0.827425128	&	0.672288795	\\
20	&	22	&	3.625232097	&	\underline{\textbf{1.518287857}}	&	\underline{1.650088133}	&	3.97000465	&	1.64832496	&	1.291540717	\\

    \hline
    \end{tabular}}
    \label{tab:my_label}
\end{table}
\hfill\break


One of the aspects we can look at is if separability is exploited by
the new communication policies. Essentially, functions 1 to 3
look at that kind of thing (Sphere, Ellipsoidal and Rastrigin). In
this case there is not a clear evidence that that happens; as a matter
of fact, the only significantly better result occurs with the Chain
MSE and Hypercube. This last one is significantly better both for 10
and 20 dimensions.

Function 9 (rotated Rosenbrock), which would give us a hint on whether
the new policies can follow a long path, does not offer a significant
variation over the original policy.

Function 10 (Ellipsoidal) is unimodal. The only significant difference
is offered by the original EvoSwarm, gut this is diluted at higher
dimensions.

The next set of functions are multimodal: F15, F17 and F18 offer
slight and not significant at higher level, advantages for the new
policies. There's a significant difference for the Chain policy at 10
dimensions, not any more at higher dimensions.

F21 and F22 are multimodal functions; in this case, Chain and
Hypercube are significantly better, although only at higher
dimensions.

In general, using these functions allowed us to characterize where
different communication policies might offer any advantage over the
baseline policy. We will try to draw some conclusions next.


\section{Conclusions}

% Discussion should include why this happens, and why there are
% differences in specific circumstances or functions. Also if there's
% some change with the problem size (dimensions) - JJ
Communication policies for multiswarm PSOs have seldom been explored
in the literature, despite the influence they might have in the final
result. This is why, in this research, we compared three communication methods between populations of
a multi-swarm system to identify an advantage resulting from these
changes: EvoSwarm, Chain and Hypercube. These three methods differ
mainly in the number of connections every population is going to
have.

However, the experiments show that difference among results obtained
by the different communication policies is not significant in most cases,
but we obtained the best results at higher dimensions. Results show a
weak evidence that restricting communications for higher number of
dimensions might be positive, over all in multimodal functions, and when the problem space is more deceptive, as is the case with function
$f_{3}$ Rastrigin Separable. In general, these are problems where the
need to keep diversity high is stronger, and restriction of
communication is a step in that direction. EvoSwarm, as a matter of
fact, does also restrict communication, so maybe the baseline for
comparison should have been a different method, using either total
restriction in communications (via totally independent swarms) or
total communication (via random or panmictic policy).

We should take
into account that in this case, we only changed the type of communication
between sub-swarms, which is only one of the policies that we can alter in this
type of system. Different communication policies might need
fine-tuning of other kind of parameters, to keep the
exploitation-exploration balance in check; this has not been done
extensively in this paper, and might be a future line of work.

In future works, other policies can be altered at the same time to see
what impact they will have. With the results obtained, we can infer
that the change to the communication method is relevant for this type
of system in specific situations. The alteration of the different work
policies in multi-swarm systems could lead to greater specialization
of these in specific scenarios. We could test, for instance, adaptive
policies, or even changing the number and size of swarms so that
exploitation can be maintained at a higher clip in some cases. These
are all lines of work that will be explored in the future.


\section*{Acknowledgments}

This paper has been supported in part by projects DeepBio (TIN2017-85727-C4-2-P)
and TecNM Project 11356.21-P.

\printbibliography


\end{document}
